\documentclass[a4paper, 12pt]{article}

%%% Meta data

	\title{Linear Algebraic Methods in Combinatorics}
 
	\usepackage{authblk}
	\author{
		Your~Name\\
		Roll~Number
	}
	\affil{
		Computer Science \& Engineering\\
		Indian Institute of Technology Palakkad
	}


%%% Page formatting

	\usepackage[margin=2cm]{geometry}
	
	\usepackage{hyperref}
	\hypersetup{colorlinks=false,linkcolor=red,citecolor=red,pdfborder={0 0 0}}

	\renewcommand{\arraystretch}{1.5}

% Math 
	\usepackage{amsmath,amsthm, amssymb}

	% Theorem environments
	\newtheorem{theorem}{Theorem}
	\newtheorem{corollary}[theorem]{Corollary}
	\newtheorem{lemma}[theorem]{Lemma}
	\newtheorem{proposition}[theorem]{Proposition}
	\newtheorem{conjecture}[theorem]{Conjecture}
	\newtheorem{observation}[theorem]{Conjecture}

	\theoremstyle{definition}
	\newtheorem{definition}[theorem]{Definition}
	\newtheorem{question}[theorem]{Question}

	\theoremstyle{remark}
	\newtheorem*{remark}{Remark}
	
	% Shorthands
	\def\bbN{\mathbb{N}}
	\def\bbZ{\mathbb{Z}}
	\def\bbQ{\mathbb{Q}}
	\def\bbR{\mathbb{R}}
	\def\bbF{\mathbb{F}}

	\def\tends{\rightarrow}
	\def\into{\rightarrow}
	\def\implies{\Rightarrow}
	\def\half{\frac{1}{2}}
	\def\quarter{\frac{1}{4}}
	
	\newcommand{\set}[1]{\left\{ #1 \right\}}
	\newcommand{\norm}[1]{\left\Vert #1 \right\Vert}
	\newcommand{\card}[1]{\left\vert #1 \right\vert}
	\newcommand{\ceil}[1]{\left\lceil #1 \right\rceil}
	\newcommand{\floor}[1]{\left\lfloor #1 \right\rfloor}

\begin{document}
\maketitle

\begin{abstract}

Summarise the article in three or four sentences. Though it appears at the top
of the article, it is better to write it after the rest of the article is
completed.

\end{abstract}

\section{Introduction}

Give a basic overview of the method in two or three paragraphs. 

\subsection{Preliminaries}

You can recall/state any basic linear algebraic terms and results (with out
proofs) that is frequently used in the article.

\begin{definition}
\label{defnVectorSpace}
	A \emph{vector space} over a field $\bbF$ is a set $V$ together with
	two operations $+ : V \times V \into V$, called \emph{vector addition},
	and $\cdot : \bbF \times V \into V$, called \emph{scalar addition} which
	satisfies the following axioms:
	\begin{enumerate}
	\item 
		$\forall u, v, w \in V,~ (u + v) + w = u + (v + w)$ 
		(Associativity of $+$), 
	\item 
		$\forall u, v \in V,~ u + v = v + u$ 
		(Commutativity of $+$), 
	\item
		There exists $0 \in V$ such that $\forall v \in V,~ v + 0 = v$
		(Identity for $+$)
	\item
		(Inverse for $+$)
	\item
		$\forall a, b \in \bbF, \forall v \in V,~ 
		a \cdot (b \cdot v) = (ab) \cdot v$
		(Compatibility of $\cdot$ and $\bbF$-multiplication)
	\item
		$\forall v \in V,~ 1 \cdot v = v$, where $1$ is the multiplicative 
		identity in $\bbF$
		(Identity for $\cdot$)	
	\item
		(Distributivity of $\cdot$ over $+$)
	\item
		(Distributivity of $\cdot$ over $\bbF$-addition)
	\end{enumerate}
\end{definition}

\begin{definition}
\label{defnLinInd}
	A set $S$ of vectors in a vector space is called \emph{linearly
	independent} if
	\[
	\sum_{v \in V} \lambda_v v = 0 \implies \forall v~ \lambda_v = 0.
	\]

\end{definition}

\begin{definition}
\label{defnDimension}
	The \emph{dimension} of a vector space is the maximum size of a set of
	linearly independent vectors in the space.
\end{definition}

\begin{theorem}
The number or linearly independent rows in any matrix is same as the
number of linearly independent columns in that matrix.
\end{theorem}

\section{Examples}

Give two or three example applications of the linear algebraic method. You can
either work out the examples formally or describe the idea behind the solution
in English.

\subsection{Example 1}

\subsection{Example 2}

\section{Difficulties}

Describe the ideas which you found difficult.

\end{document}
